\section{Introduction}

Computer games is a big business. Competitive gaming growing the interest in this business. The competitive gaming events, the so-called electronic sports (sPorts). State-of-the-art multi-player online batle arena (MOBA) games are the most popular and succesful games in this regard. Game tournaments always give significant awards with big prize of money. Mostly, professional player can make a living from the prize. 

In this paper we analyzing the player roles using supervised machine learning (ML) , investigate the applicability of feature normalization to enhance accuracy of prediction results. We also comparing our result with previous paper. Christoph Eggert on his paper said that their most stable and performing classifier is logistic regression. We will use the same classification method with their work. The different is we will using ROpenDota to built the data set.

An approach to ML in computer games in general was proposed by Drachen et al.\cite{drachen2014skill} . They suggest using unsupervised learning algorithms, specifically k-means and \textit{Simplex Volume Maximization}, to cluster player behavior data.

Our investigation about applicability and performance of \textit{supervised machine learning} (ML) to classify player roles based on behavior in Dota 2, the most popular MOBA game.

This paper contributes to the state of the art in several ways: We provide an in-depth discussion and novel approaches regarding the construction of complex attributes from low level data extracted from Dota 2 replay files, together with an evaluation of these attributes with respect to different classifiers. Based on the resulting reduced set of attributes, we compare and discuss the performance of a range of supervised classification algorithms, including logistic regression, random forest decision trees, support vector machines (in combination with Sequential Minimal Optimization), naive Bayes and Bayesian networks, classifying both with a newly established larger set of player roles, as well as with a reduced set inspired by related work ~\cite{eggert2015classification}.