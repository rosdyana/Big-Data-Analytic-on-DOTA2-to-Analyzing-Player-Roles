\section{Results and Conclusion}
\label{sec:conclusion}

\subsection{Result}
\label{sec:results}

We trained and evaluated several different classifiers using 10-fold cross-validation and independent test on our data set. The classifiers were based on existing works and complemented by commonly used classification approaches. It included: \textit{Random forest decision trees}, \textit{logistic regression}, \textit{support vector machines with sequential minimal optimization}, \textit{naive bayes classifier}, and \textit{bayesian networks}. We are using WEKA library and tool for the technical platform. The classifiers were evaluated according to several established performance metric (accuracy, mean absolute error (MAE) \cite{willmott2005advantages}, and area under ROC (AUC) \cite{huang2005using}.

\renewcommand{\arraystretch}{1.5}
\setlength{\tabcolsep}{8pt}
\begin{table}[]
\centering
\caption{Summary of 10-fold cross -validation and independent test accuracies with feature normalization data set.}
\label{table:table2}
\resizebox{0.5\textwidth}{!}{%
\begin{tabular}{|l|l|l|l|l|}
\hline
\multicolumn{1}{|c|}{\textbf{Full Classes}}    & \multicolumn{2}{c|}{\textbf{10-Folds Cross Validation}} & \multicolumn{2}{c|}{\textbf{Independent Test}} \\ \hline
Original                                       & Random forest                 & 89.37\%                 & Random forest                & 88.54\%         \\ \hline
Linear Scale                                   & Random forest                 & 89.77\%                 & Logistic Regression          & 81.65\%         \\ \hline
Min-Max Normalization                          & Random forest                 & 89.53\%                 & Logistic Regression          & 81.65\%         \\ \hline
Softmax Scaling                                & Random forest                 & 80.13\%                 & Random forest                & 79.81\%         \\ \hline
Z-Score Normalization                          & Random forest                 & 89.68\%                 & Random forest                & 86.52\%         \\ \hline
\multicolumn{1}{|c|}{\textbf{Reduced Classes}} & \multicolumn{2}{c|}{\textbf{10-Folds Cross Validation}} & \multicolumn{2}{c|}{\textbf{Independent Test}} \\ \hline
Original                                       & Random forest                 & 94.52\%                 & Random forest                & 93.53\%         \\ \hline
Linear Scale                                   & Random forest                 & 94.33\%                 & Random forest                & 88.48\%         \\ \hline
Min-Max Normalization                          & Random forest                 & 94.39\%                 & Random forest                & 88.78\%         \\ \hline
Softmax Scaling                                & Random forest                 & 92.59\%                 & Random forest                & 92.34\%         \\ \hline
Z-Score Normalization                          & Random forest                 & 89.40\%                 & Random forest                & 92.10\%         \\ \hline
\end{tabular}%
}
\end{table}

Table \ref{table:table2} list the highest accuracy from 10-fold cross-validation and independent test of our works. Random forest is the most stable with good performance from our works. The accuracy result also reasonable. it means the good accuracy is not overfitting after checking the cross-validation with the independent test results.

\renewcommand{\arraystretch}{1.5}
\setlength{\tabcolsep}{8pt}
\begin{table}[]
\centering
\caption{Comparison with previous works.}
\label{table:table3}
\resizebox{0.5\textwidth}{!}{%
\begin{tabular}{|c|c|c|c|c|c|c|}
\hline
\textbf{Classifier}             & \multicolumn{2}{c|}{\textbf{Accuracy}} & \multicolumn{2}{c|}{\textbf{Mean Absolute Error}} & \multicolumn{2}{c|}{\textbf{Wgt. Avg. AUC}} \\ \hline
\textbf{Full set of classes}    & *                  & **                & *                       & **                      & *                    & **                   \\ \hline
Random Forest                   & 76.27\%            & 89.37\%           & 0.0905                  & 0.0691                  & 0.943                & 0.984                \\ \hline
Logistic Regression             & 75.85\%            & 82.64\%           & 0.0826                  & 0.1013                  & 0.947                & 0.962                \\ \hline
SMO                             & 75.28\%            & 81.65\%           & 0.1753                  & 0.2503                  & 0.926                & 0.91                 \\ \hline
Bayesian Networks               & 72.03\%            & 70.54\%           & 0.0801                  & 0.122                   & 0.933                & 0.927                \\ \hline
Naive Bayes                     & 70.76\%            & 74.43\%           & 0.0769                  & 0.1055                  & 0.933                & 0.95                 \\ \hline
\textbf{Reduced set of classes} & *                  & **                & *                       & **                      & *                    & **                   \\ \hline
Bayesian Networks               & 96.58\%            & 92.61\%           & 0.0322                  & 0.0919                  & 0.995                & 0.973                \\ \hline
SMO                             & 96.15\%            & 88.15\%           & 0.2308                  & 0.1185                  & 0.975                & 0.875                \\ \hline
Logistic Regression             & 96.15\%            & 88.15\%           & 0.0381                  & 0.1662                  & 0.993                & 0.953                \\ \hline
Naive Bayes                     & 95.58\%            & 89.84\%           & 0.0383                  & 0.2319                  & 0.994                & 0.89                 \\ \hline
Random Forest                   & 91.17\%            & 94.52\%           & 0.1182                  & 0.0916                  & 0.985                & 0.987                \\ \hline
\end{tabular}%
}
\end{table}

Table \ref{table:table3}, the comparison result of our works with previous works. Which is our works is quite better in full set classes. Random forest still dominating for the accuracy result.

Classify the player roles is not easy. We asking \textit{Dota 2} experts to manually classify the player roles. The responses were highly divergent among them. This illustrates that the classification tasks is difficult, even for human experts. Our data set from the professional tournament (Kiev Major 2017 and \textit{The International 2016}) with 8419 data set, We separated into 70\% for training data and 30\% the rest for testing data. We also classified our data set with the same attributes but with a reduced set of classes (\textit{carry}, \textit{support}) inspired by Gao et al. \cite{gao2013classifying}, classification with our attributes achieved a higher accuracy for our data set. A direct comparison is not possible since the data sets differ. Still, the result indicate that classification with a reduced set of classes works well and could already be employed for many applications. Summarizing, we can state that the full set of classes shows promising results but also highlights that, although these classes are accepted by \textit{Dota 2} experts, some are ambiguous even to humans manually labeling the data.

\subsection{Conclusion and Future Work}
\label{sec:conclusion}

Not yet :v
